\section{Xây dựng hệ thống AI}

Hệ thống \textbf{InsightTune} có tích hợp hai dịch vụ (service) AI, hai dịch vụ này được triển khai sao cho tương thích tốt với toàn bộ
hệ thống và có tính đặc trưng cao (phân biệt với những dịch vụ AI cho đa tác vụ như: Gemini hay ChatGPT). Chúng tôi xin trình bày
những kiến thức nền tảng để xây dựng hiệu quả hai dịch vụ này qua hai phần riêng biệt dưới đây: dịch vụ gợi ý và trợ lý ảo.

\subsection{Dịch vụ gợi ý}
Để xây được một dịch vụ gợi ý tốt trong một thống lớn của \textbf{InsightTune}, ta cần hiểu hai miền kiến thức chính:
\begin{itemize}
    \item Một là về kiến trúc hệ thống, để từ đó hiểu được đầu vào và đầu ra của dịch vụ cũng như phương thức giao tiếp với những tác vụ khác của hệ thống.
    \item Hai là về xây dựng và triển khai một mô hình gợi ý.
\end{itemize}
Những yêu cầu kiến thức về kiến trúc hệ thống đã được trình bày tại \ref{sec:arch_knowledge} nên ở phần này chúng tôi chỉ tập trung vào giải thích ngắn
gọn những kiến thức xây dựng và triển khai mô hình gợi ý.

\subsubsection{Xây dựng}
Xây dựng mô hình gợi ý tức là chuẩn bị tập dữ liệu mẫu chia ra làm tập huấn luyện và tập kiểm tra, sau khi hoàn tất quá trình huấn luyện
trên tập huấn luyện, ta thu được một mô hình hoạt động tốt trên tập kiểm tra.

Kiến thức cần có để xây dựng lại một mô hình gợi ý như chúng tôi làm đó là:
\begin{itemize}
    \item Kiến thức về học tăng cường
    \item Xử lý dữ liệu
    \item Các thư viện học máy trong Python: Pytorch, numpy, pandas
    \item Công cụ đo lường và biểu diễn: wandb
\end{itemize}

\subsubsection{Triển khai}
Sau khi có được một mô hình gợi ý, chúng tôi triển khai mô hình như một dịch vụ trong hệ thống. Điều này phải tương thích với yêu cầu đầu vào và đầu ra của 
dịch vụ.

Những kiến thức cần có để triển khai hiệu quả là:
\begin{itemize}
    \item Kiến thức xây dựng dịch vụ bằng FastAPI
    \item Thiết lập đóng gọi dịch vụ bằng Docker
\end{itemize}

\subsection{Trợ lý ảo}