\setcounter{section}{1}
\section{Biểu đồ tuần tự}

\subsection{Chatbot}
\label{sec:sequence_chatbot}
Supervisor trực tiếp nhận câu hỏi từ chatbot system, phân tích (áp dụng
Chain of Thinking). Khi nhận được câu hỏi, hệ thống gửi cho Supervisor, 
Supervisor sẽ quyết định đây có yêu cầu query database
hay không. Nếu có, định tuyến tới Querier, ngược lại tới Researcher (search agent).

Querier làm các bước sau:
\begin{itemize}
    \item Sinh câu lệnh SQL, yêu cầu Verifier xác nhận (1)
    \item Nếu không có lỗi, gửi câu lệnh SQL tới database, nếu có lỗi
    lặp lại từ (1) (2)
    \item Nhận kết quả từ database (3)
    \item Yêu cầu Evaluator đánh giá kết quả (4)
    \item Nhận kết quả đánh giá từ Evaluator, đạt gửi tới Supervisor,
    nếu không lặp lại từ bước (1) (5)
\end{itemize}

Researcher làm các bước sau:
\begin{itemize}
    \item Nhận câu hỏi từ Supervisor (1)
    \item Gọi Tavily để tìm kiếm thông tin (2)
    \item Nhận kết quả từ Tavily (3)
    \item Reformat và gửi trả lại cho Supervisor (4)
\end{itemize}
Sau đó, Supervisor sẽ tổng hợp kết quả,
đưa ra câu trả lời cho người dùng. 
Hình trang tiếp theo mô tả quá trình này.
\newpage
\begin{figure}[H]
  \centering
  \includegraphics[width=1\textwidth]{figures/chatbot.png}
  \caption{Biểu đồ tuần tự cho chatbot}
\end{figure}
\subsection{Chỉnh sửa giao dịch}
\begin{itemize}
	\item User điền thông tin giao dịch cần chỉnh sửa.
	\item Giao diện người dùng (UI) đóng gói dữ liệu thành DTO (Data Transfer Object) và gửi đến Controller.
	\item TransactionController nhận DTO và thực hiện validate.
	\item Nếu DTO hợp lệ, controller gọi service để cập nhật.
	\item TransactionService nhận yêu cầu và validate Entity.
	\item Nếu Entity hợp lệ, UI hiển thị thông báo thành công. Nếu Entity không hợp lệ, UI hiển thị thông báo lỗi.
\end{itemize}
\newpage
\begin{figure}[H]
	\centering
	\includegraphics[width=1\textwidth]{figures/edit-transac-sq.png}
	\caption{Biểu đồ tuần tự cho chức năng chỉnh sửa giao dịch}
\end{figure}

\subsection{Xem báo cáo chi tiêu}
\begin{itemize}
	\item Người dùng yêu cầu báo cáo.
	\item Chọn thời gian (vd: tháng 7/2023).
	\item Hệ thống truy vấn database để lấy tất cả giao dịch trong khoảng thời gian người dùng chọn.
	\item Nếu không có giao dịch nào → Trả về biểu đồ trống
	\item Nếu có giao dịch → Tiếp tục xử lý.
	\item Phân tích dữ liệu: Gộp các giao dịch cùng loại, Tính tổng tiền cho từng danh mục -> Xây dựng biểu đồ thống kê.
\end{itemize}
\newpage
\begin{figure}[H]
	\centering
	\includegraphics[width=1\textwidth]{figures/view-report-sq.png}
	\caption{Biểu đồ tuần tự cho chức năng xem báo cáo chi tiêu}
	\end{figure}

\subsection{Thêm giao dịch mới}
\begin{itemize}
	\item Người dùng thêm giao dịch: nhấn nút ``Thêm giao dịch'' trên web, Điền thông tin (số tiền, danh mục, ghi chú...).
	\item Hệ thống xử lý: Kiểm tra dữ liệu, nếu dữ liệu hợp lệ lưu vào database.
	\item Hệ thống hiển thị kết quả.
\end{itemize}
\newpage
\begin{figure}[H]
	\centering
	\includegraphics[width=1\textwidth]{figures/add-transac-sq.png}
	\caption{Biểu đồ tuần tự cho chức năng thêm giao dịch mới}
\end{figure}
