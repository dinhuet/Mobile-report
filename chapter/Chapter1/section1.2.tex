\section{Các hệ thống tương tự}

Những phần mềm nghe nhạc hiện nay rất đa dạng và phong phú, từ những ứng dụng phổ biến như Spotify, Apple Music, YouTube Music, Zing MP3, SoundCloud cho đến những ứng dụng mã nguồn mở như
SpoTube. Mỗi ứng dụng đều có những đặc điểm nổi bật và nhược điểm riêng. Để đưa cho người đọc có cái nhìn tổng quan về các hệ thống phát nhạc hiện hành, chúng 
tôi so sánh một số hệ thống phát nhạc phổ biến hiện nay trong bảng \ref{tab:compare_music_systems}. Qua việc xem xét những nhược điểm này, chúng tôi nhận thấy
rằng việc phát triển một dự án mã nguồn mở tích hợp AI sẽ mang lại nhiều lợi ích cho cộng đồng người dùng và các nhà phát triển. Điều này là động lực chính để
chúng tôi phát triển \textbf{InsightTune}, một phần mềm với mã nguồn đơn giản, dễ dàng tùy chỉnh để làm framework ví dụ về việc tích hợp AI trong một ứng
dụng nghe nhạc hiện đại.

\begin{table}[H]
\centering
\begin{tabular}{|p{4cm}|p{4cm}|p{4cm}|}
    \hline
    \textbf{Tên hệ thống} & \textbf{Đặc điểm nổi bật} & \textbf{Nhược điểm} \\
    \hline
    Spotify &
    Giao diện thân thiện, đa nền tảng, hệ thống gợi ý bài hát tốt &
    Phần mềm mã nguồn đóng, chủ yếu hoạt động theo mô hình thuê bao, không có chat bot trợ lý ảo cá nhân \\
    \hline
    YouTube Music &
    Thư viện bài hát và video rất lớn, tích hợp chặt với tài khoản Google, hỗ trợ cá nhân hoá playlist &
    Nhiều quảng cáo ở bản miễn phí, phụ thuộc vào tài khoản Google, không phải mã nguồn mở nên khó tuỳ chỉnh \\
    \hline
    Apple Music &
    Chất lượng âm thanh cao, tích hợp tốt với hệ sinh thái thiết bị Apple, hỗ trợ lossless/spatial audio &
    Chủ yếu hỗ trợ tốt trên thiết bị Apple, yêu cầu trả phí theo tháng, phần mềm mã nguồn đóng \\
    \hline
    Zing MP3 &
    Phổ biến tại Việt Nam, thư viện nhạc Việt phong phú, hỗ trợ nghe offline &
    Tập trung nhiều vào nội dung trong nước, có quảng cáo trên bản miễn phí, không hỗ trợ tuỳ chỉnh sâu hệ thống \\
    \hline
    SoundCloud &
    Cộng đồng nghệ sĩ độc lập lớn, dễ dàng upload và chia sẻ bản nhạc cá nhân &
    Chất lượng và độ ổn định nội dung không đồng đều, ít tính năng gợi ý thông minh so với các nền tảng lớn \\
    \hline
    SpoTube &
    Miễn phí, kết hợp được với YouTube &
    Giao diện chưa thân thiện, chưa có hệ thống gợi ý bài hát \\
    \hline
\end{tabular}
\caption{So sánh các hệ thống phát nhạc hiện hành}
\label{tab:compare_music_systems}
\end{table}

\clearpage